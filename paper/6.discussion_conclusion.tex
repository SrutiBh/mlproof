\section{Discussion and Conclusion}

Automatic cell boundary segmentation is difficult, and trying to improve such segmentations automatically as a post-process through split and error correction is, in principle, no different than trying to improve the underlying cell boundary segmentation. This is shown by the approximately equivalent VI distributions of the initial segmentation and our automatic segmentation correction (Fig.~\ref{fig:results}). Due to the task difficulty, manual proofreading of connectomics segmentations is necessary, but it is time consuming and error prone, as can be seen from the Dojo human trials: on average, participants made the segmentations worse. However, there is value in being able to recommend to users possible regions for correction, as the time cost of proofreading is dominated by the visual search for errors.

We have addressed this problem through training a CNN to detect ambiguous regions from labeled data---in effect, (re-)learning a confidence measure on boundaries. This allows us to identify split and merge errors, and also to recommend their corrections, which is an improvement over existing systems which just provide semi-automatic merge error correction. Our experiments have shown that guided proofreading has the potential to reduce VI over existing interactive proofreading tools. This helps reduce the proofreading bottleneck \change{in} the analysis of large connectomics datasets. To encourage testing of our proposed architecture on more data, we provide the trained networks and classifier code as free and open source software at (link omitted for review).

% with different error rates

%\JT{Do the edges we correct come from areas where RhoANA is itself less confident? Have we just re-learned a confidence measure on RhoANA?}

%Currently it is a significant bottleneck in the analysis of large data sets in connectomics. In this paper we propose a combination of machine learning and minimal user guidance to perform segmentation corrections. Our results show that this combination makes it possible to enhance the performance as well as to decrease the required time to perform segmentation corrections. We would like to encourage testing of our proposed architecture on more data and provide the developed software and trained networks free and open source at (link omitted for review).%\footnote{The implementation, trained networks and supplementary material are available at \url{http://rhoana.org/cnnproofreading}}.

%Although, our system was not able to provide a reliable fully automatic proofreading solution, the interactive results are promising and we believe that our work leads us towards a fully automatic system. 


%The performances of the trained networks were limited based on the lack of ground truth data matching our automatic segmentation pipeline. For baseline comparison, we chose a recently published user study comparing interactive proofreading tools \cite{haehn_dojo_2014}. This comparison makes sense but unfortunately the used dataset is very small. Furthermore, we would like to have non-experts and experts test our proofreading application with minimal user input.


