\section{Application}

In our experiments, we observed the best performance using a combination of user guidance and our trained network. In contrast to fully interactive proofreading tools like Dojo, Mojo\change{,} and Raveler, our system requires only minimal user input. We distinguish between merge and split errors and provide a very simple user interface to correct them (Fig. \ref{fig:prototype}).
The system shows only one potential error \change{in the interface: either a potential false merge or split}. In the case of merge errors, the user sees the five highest-scoring possible boundaries as overlays on the corresponding grayscale image, and can also draw a new boundary interactively. The user then chooses one of the suggestions, draws a boundary, or marks the cell as correct. For split errors, the system shows the grayscale image and a possible border, and the user marks whether the cell is correct. Our Dojo user study experiment baseline was limited to 30 minutes, and participants performed 59 corrections in average ($\approx30$ seconds per correction). Our experiments suggest that even non-experts can perform a correction using our system in $<5$ seconds, resulting in increased proofreading \change{throughput}.